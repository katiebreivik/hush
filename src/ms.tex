% Define document class
\documentclass[twocolumn]{aastex631}

\newcommand{\vdag}{(v)^\dagger}
\newcommand\aastex{AAS\TeX}
\newcommand\latex{La\TeX}
\usepackage{amsmath}
\newcommand\numberthis{\addtocounter{equation}{1}\tag{\theequation}}
\newcommand{\cosmic}{\texttt{COSMIC}}
\newcommand{\legwork}{\texttt{LEGWORK}}

\shorttitle{Applying a metallicity-dependent binary fraction to DWD formation}
\shortauthors{Thiele, Breivik, \& Sanderson}

% Begin!
\begin{document}

% Title
\title{Applying the metallicity-dependent binary fraction to double white dwarf formation: \\ Implications for LISA}

% Author list
\correspondingauthor{Sarah Thiele}
\email{sarahgthiele@gmail.com}

\author[0000-0001-7442-6926]{Sarah Thiele}
\affiliation{Department of Physics and Astronomy, University of British Columbia, 6224 Agricultural Road, Vancouver, BC, V6T 1Z1, Canada}
\affiliation{Canadian Institute for Theoretical Astrophysics, University
of Toronto, 60 St. George Street, Toronto, Ontario, M5S 1A7,
Canada}
\author[0000-0001-5228-6598]{Katelyn Breivik}
\affiliation{Center for Computational Astrophysics, Flatiron Institute, 162 Fifth Ave, New York, NY, 10010, USA}
\affiliation{Canadian Institute for Theoretical Astrophysics, University
of Toronto, 60 St. George Street, Toronto, Ontario, M5S 1A7,
Canada}
\author[0000-0003-3939-3297]{Robyn E. Sanderson}
\affiliation{Department of Physics and Astronomy, University of Pennsylvania, 209 South 33rd Street, Philadelphia, PA 19104, USA}
\affiliation{Center for Computational Astrophysics, Flatiron Institute, 162 Fifth Ave, New York, NY, 10010, USA}

% Abstract with filler text
\begin{abstract}
   Short-period double white dwarf (DWD) binaries will be the most prolific source of gravitational waves (GWs) for the Laser Interferometer Space Antenna (LISA). Not only will tens of thousands of DWDs be individually resolved, but DWDs with GW frequencies below $\sim1\,\rm{mHz}$ will be the dominant contributor to a stochastic foreground caused by confusion from overlapping GW signals, limiting the detectability of individual sources of all kinds. The number of DWDs in the Milky Way strongly depends on the Galactic stellar binary fraction. Population modelling of Galactic DWDs typically assumes a standard binary fraction of $50\%$. However, recent observations have shown that the binary fraction of close ($P_{\text{orb}}$ $\leq 10^4$ days) solar-type stars exhibits a strong anti-correlation with metallicity. In this study we perform the first simulation of the Galactic DWD population observable by LISA which incorporates an empirically-derived metallicity-dependent binary fraction. We simulate DWDs using the binary population synthesis suite \cosmic\ and incorporate a metallicity-dependent star formation history to create a Galactic population of short-period DWDs. We compare two models: one which assumes a metallicity-dependent binary fraction, and one with a binary fraction of 50\%. We find that while metallicity impacts the evolution and intrinsic properties of our simulated DWD progenitor binaries, the LISA-resolvable populations of the two models remain roughly indistinguishable. However, the size of the total Galactic DWD population orbiting in the LISA frequency band is reduced by more than half when accounting for a metallicity-dependent binary fraction. This effect serves to lower the confusion foreground, effectively increasing the sensitivity for detecting all types of low-frequency LISA sources. We repeat our analysis for three different assumptions for Roche-lobe overflow interactions and find the population reduction to be robust when a metallicity-dependent binary fraction is assumed.

\end{abstract}

% Keywords
\keywords{Binary stars --- Stellar evolution --- GW astronomy}

\section{Introduction} \label{sec:intro}
Most stars in the Galaxy will end their lives as white dwarfs. Of the stars which are born with a binary companion, many will undergo interactions which bring the two stars closer together, eventually forming a close double white dwarf (DWD). Close DWDs, with orbital periods shorter than $\sim2.75\,\rm{hr}$ %\snotes{[close DWDs are defined to have $P_{\rm{orb}}\leq10^4$ days - are we noting 2.75 hrs for GW sources specifically? If so maybe we should reword this bit because it reads like it conflicts with the abstract?]},
%Katie: Close DWDs is a different classification than close binaries.
are the largest source by number of mHz gravitational waves (GWs) in the Galaxy \citep[e.g.,][]{LISAMissionProposal}. The Laser Interferometer Space Antenna (LISA) is expected to resolve at least $10^4$ individual DWD binaries in the Milky Way and will also observe GW emission from the entire Galactic DWD population through the unresolved foreground created by overlapping signals at sub-mHz frequencies \citep[e.g.,][]{Nelemans2001a, Ruiter2010, Nissanke2012, Yu2013, Korol2017, Lamberts2019, Breivik2020a}. The resolved population will enable the study of several important aspects of binary evolution like the strength of tides \citep{Valsecchi2012} and the stability of mass transfer in DWD systems \citep[e.g.,][]{Marsh2004, Shen2015, Gokhale2007, Sepinsky2014, Kremer2015} as well as provide a probe of Galactic structure \citep{Korol2019} and the Local Group \citep{Korol2018}. The shape and strength of the Galactic DWD foreground can also be used as a tool to study the structure of the Milky Way \citep{Benacquista2006, Breivik2020b} as well as the separation distribution of close DWDs \citep{Korol2021}.

When viewed strictly as a source of noise, the unresolved Galactic DWD foreground is the dominant noise source for LISA in the sub-mHz part of LISA's frequency range. This extra noise above the detector noise floor affects the detection of all other LISA sources including extreme mass ratio inspirals \citep[e.g.,][]{Berti2006, Barack2007, Babak2017, Moore2017}, merging black holes with masses between $10^4$--$10^7\,M_{\odot}$ \citep[e.g.,][]{Klein2016, Bellovary2019}, and cosmological GW backgrounds \citep[e.g.,][]{Bartolo2016, Caprini2016, Caldwell2019}. For sources which have signals buried by the Galactic DWD population, the foreground must be carefully analyzed and subtracted \citep[][]{Adams2014,Cornish2020,Littenberg2020,Boileau2021}. The number of resolved DWDs and the height of the unresolved DWD foreground are a direct consequence of the number of DWD progenitors which form and evolve over the Milky Way's history. 

While the binary fraction remains approximately constant across a large metallicity range ($-1.5 \leq$ [Fe/H] $< 0.5$) for wide binaries, close OB stars, and the stellar Initial Mass Function (IMF) \citep{Moe2017, Moe2019}, the binary fraction for solar-type star systems with orbital period $P_{\text{orb}}\leq 10^4$ days (separation $a \leq 10$ AU) shows a strong anti-correlation with metallicity \citep[e.g.][]{Badenes2018, Moe2019, Mazzola2020, Price-Whelan2020}. Because close DWDs are the remnants of close, solar-type binary stars, this anti-correlation plays an important role in the formation, evolution, and characteristics of the DWD population that LISA will observe.

%These relatively close binaries are likely to form from fragmentation, accretion, and inward migration in protostellar disks. This is in contrast to wide binaries whose formation channels stem from fragmentation of molecular cores, which is generally unaffected by the opacity of the material. 
%This is depicted in the ``metallicity invariance" for the IMF and binary fraction beyond separations of 200 au for -1.0 $<$ [Fe/H] $<$ 0.5 of Mazzola et al. (2020). Within protostellar disks however, it has been shown through analytical models and hydrodynamic simulations that the probability of disk fragmentation decreases with increasing metallicity (Mazzola et al. 2020). Metal-poor cores are unable to radiate through molecular transitions as effectively, and so are generally hotter and must reach higher masses in order to collapse into disks, which instigates higher accretion rates and results in gravitational instability and thus fragmentation. Furthermore, if you decrease a disk’s metallicity and thus its optical depth, the mid-plane of the disk is able to radiate and cool more efficiently which encourages fragmentation. (Mazzola et al. 2020 and others).

%Numerically, the dominant sources for LISA will be DWDs from the Milky Way. There is an expectation that several tens of millions of detached DWDs will be present in the LISA band with about ten thousand of these to be individually resolvable  with $f_\text{GW} \geq 0.4$ mHz (Nelemans et al. 2001b). DWDs are nice systems to work with: a compact binary which emits continuously and near monochromatically GW signals in the source frame (LISA proposal REF), and of which the second harmonic of their signals can be measured with high accuracy. In general, higher frequency systems are louder and better characterized than low frequency counterparts. In fact, at low frequencies we find that the high number of DWDs begin to overlap in their signals, creating a stochastic background. This effective “confusion signal” arises for these low frequencies from having more than one DWD per frequency bin, $f_\text{bin} = 1 / t_\text{obs}$, where $t_\text{obs}$ is the length of observation time for the LISA mission, which we assume to be four years (LISA proposal REF). 

%The science that the LISA mission will contribute to is diverse in scope. Specific to DWDs, the hope is to use the distribution of chirp masses and periods to constrain the impact of the common envelope, which drastically tightens the orbit of the systems (Toonen et al. 2014). A complete sample will also allow for direct comparison with the post-common envelope binaries, which have undergone only one episode of the common envelope (Rebassa-Mansergas et al. 2012). In some cases, the frequency derivative will be measurable and allow us to determine if mass transfer is happening (Breivik et al. 2018) and/or tidal interactions are deforming the white dwarfs.

%\subsection{The metallicity-dependent binary fraction}
To date, population synthesis studies of the Galactic population of close DWDs have either assumed a $100\%$ binary fraction or a $50\%$ binary fraction, such that for every three stars formed, two reside in a binary system \citep{Nelemans2001a, Yu2013, Korol2017, Lamberts2019}. In this study, we investigate the effects of a metallicity-dependent binary fraction on the formation and evolution of DWDs. To this end, we create synthetic present-day Milky Way-like galaxies of DWDs and specifically select systems with GW signals that may be observable by the space-based detector LISA. Throughout, we make comparisons between the standard assumption of a constant $50\%$ initial binary fraction (hereafter model F50) and one with a metallicity-dependent binary fraction (hereafter model FZ).  

In Section \ref{sec:simulations} we discuss our binary evolution assumptions used in simulating DWD populations and detail the process to produce present-day synthetic Milky-Way-like galaxies. In Section \ref{sec:LISA_obs} we review the derivation of LISA detectability for circular DWD populations at mHz frequencies. In Section \ref{sec:results} we detail results showing how a metallicity-dependent binary fraction affects the formation and evolution of DWD populations assuming a fiducial set of binary evolution assumptions. In Section \ref{sec:LISA_met} we detail the metallicity dependence of the LISA DWD foreground and resolved population. Finally, we repeat our analysis for three sets of binary evolution assumptions to show a robust reduction of the size of the Galactic DWD foreground in LISA for populations simulated with a metallicity-dependent binary fraction in Section \ref{sec:model_compare} and conclude in Section \ref{sec:conclusions}.  

\section{Simulating a Galactic DWD population}\label{sec:simulations}
In this section we describe the setup of our DWD simulations using the binary population synthesis suite \cosmic, and the process to scale these simulations to create Milky Way-like galaxies using the star formation history of galaxy \textbf{\textbf{m12i}} in the Latte suite of the FIRE-2 simulations \citep{Wetzel2016, Hopkins2018} and stellar positions assigned according to the the Ananke framework \citep{Sanderson2020}. 

%The Latte suite of FIRE-2 cosmological zoom-in baryonic simulations of Milky Way-mass galaxies \citep{Wetzel2016}, part of the Feedback In Realistic Environments (FIRE) simulation project, were run using the Gizmo gravity plus hydrodynamics code in meshless finite-mass (MFM) mode \citep{Hopkins2015} and the FIRE-2 physics model \citep{Hopkins2018}. Synthetic Gaia DR2-like surveys of the Latte suite of FIRE-2 simulations were created via the Ananke framework \citep{Sanderson2020}.

\subsection{Binary population models}
\label{sec:bin_pop}
We simulate the evolution of DWD progenitor populations using \cosmic\footnote{https://cosmic-popsynth.github.io}, an open-source Python-based rapid binary population synthesis suite which employs single and binary star evolution using \texttt{SSE/BSE} \citep{Hurley2000, Hurley2002}. Several modifications have been added to \cosmic\ which incorporate updates for massive star evolution and binary interactions. For a detailed description of these modifications see \citet{Breivik2020a}. \cosmic\ has been used in several studies to examine the effects of binary evolution on binary populations from blue stragglers \citep{Leiner2021} and heartbeat stars \citep{Jayasinghe2021}, to white dwarf populations \citep{Kremer2017,Breivik2018,Kilic2021}, to merging compact object populations in isolated binaries \citep{Zevin2020b, Zevin2020a, Zevin2021, Wong2021, Mandhai2021} and in dynamical environments around super-massive black holes \citep{Stephan2019, Wang2021}. 
%Recently, \cosmic\ was integrated with the Cluster Monte Carlo (CMC) code to enable direct comparisons between isolated binary evolution and the evolution of stars in globular clusters \citep{Rodriguez2021}. Since this work considers the effects of the metallicity-dependence of the close isolated binary fraction, we leave direct comparisons to globular clusters to future work.

\cosmic\ is especially useful for efficient generation of large populations of compact binaries. Instead of choosing a fixed number of binary stars for each simulation, \cosmic\ iteratively simulates populations until parameter distributions of the binary population converge to a stable shape as more binaries are added. This process is quantified through the $match$ parameter inspired by matched filtering techniques \citep[e.g. Eq. 6 of ][]{Chatziioannou2017} defined as

\begin{equation}
   match = \frac{\sum_{k=1}^{N}P_{k,i}P_{k,i+1}}{\sqrt{\sum_{k=1}^{N}P_{k,i}P_{k,i}\sum_{k=1}^{N}P_{k,i+1}P_{k,i+1}}},
\end{equation}
where $P_{k,i}$ represents the height of bin $k$ on the $i\rm{th}$ iteration \citep{Breivik2020a}. In this study, we simulate binaries until $\log_{10}(1-match) \leq -5$ for the masses and orbital periods of each DWD population at the formation of the second WD. Since all DWD progenitor binaries simulated with \cosmic\ are circularized through mass transfer or tides before the second WD forms \citep[e.g.][]{Marsh2004, Gokhale2007, Sepinsky2014, Kremer2015}, we do not consider convergence of DWD eccentricities.

% I'm suggesting text above since we have a boiler-plate style that we use to stay consisten in the literature.
%Rather than evolving an initial binary population’s individual stars - a computationally expensive process - \cosmic\ uses Monte Carlo statistics to create fixed populations that would be expected based on parameters like SFH and the known processes within binary star evolution. In the \cosmic\ framework, binaries are initialized iteratively with zero age main sequence (ZAMS) binary parameters and tested against the final parameter distributions desired by the user until they match. The output of \cosmic\ contains enough information that a user can analyze both individual systems of interest as well as the overall population, from ZAMS until compact binary formation. 

%In this study, we used \cosmic\ 's cosmic-pop package to create fixed populations of DWDs, keeping all parameters except DWD type and metallicity Z constant. The output fully describes the binary parameter distributions that result from our SFH and binary evolution model, coined BSEDict, which is specified in an inifile containing sections on filters, convergence, sampling, and BSE flags, which will be described below:

%\katie{Based on the restructuring I did, I think you should expand this paragraph to describe (1) since we are interested in a metallicity-dependent binary fraction, we simulate metallicity bins separately, (2) since the masses and orbital periods of each DWD type at formation span a wide range, we simulate them separately.}
%For a given metallicity, a fixed population of one million binaries was created for each DWD subtype (discussed in Section \ref{sec:ini}). 
%\katie{Is this true? I think we probably simulated more. We should talk about this today!}

%moved paragraph up here from below since it takes care of some of the things I suggested in the paragraph above!

The masses and orbital periods at the formation of the second-formed WD span a wide range depending on the WD binary component types, thus we consider four DWD combinations: two helium WDs (He + He), a carbon-oxygen WD orbiting a helium WDs (CO + He), two carbon-oxygen WDs (CO + CO), and an oxygen-neon orbiting helium, carbon-oxygen, or oxygen-neon WD (ONe + X). For each DWD type we simulate a grid of $15$ metallicities spaced uniformly in $\log_{10}(Z)$ between $Z=10^{-4}$ to $0.03$, to account for the limits of the \citet{Hurley2000} stellar evolution tracks employed in \cosmic. This results in a total of $60$ populations across all DWD types and metallicities. The output of \cosmic\ contains information limited to intrinsic binary properties like mass and orbital period. External parameters like Galactic position and orientation are assigned in a post-processing scheme which uses metallicity-dependent positions and ages from the Ananke framework of galaxy \textbf{m12i} from the Latte Suite of the FIRE simulations (see Section~\ref{sec:FIRE} for details).

%this is covered in my text above.
%We iteratively simulate binary populations until the distributions of the DWD masses and orbital period at the formation of the second-formed WD converge to a stable shape. Eccentricity does not need to be analyzed, as most DWDs are circularized by tidal forces (Marsh et al. 2004, Gokhale et al. 2007, Kremer et al. 2015/17), which also simplifies many calculations later on for GW evolution and GW signals.

We assume that the Zero Age Main Sequence (ZAMS) masses, orbital periods and eccentricities for each binary are independently distributed. We choose primary masses following \citet{Kroupa2001}, a flat mass ratio distribution \citep{Mazeh1992, Goldberg1994}, a log-uniform period distribution following Opik's Law, and a uniform eccentricity distribution following \cite{Geller2019}. We initialize all binaries with the same evolution time of $13.7\,\rm{Gyr}$ to capture all potential evolution within a Hubble time. We assume a $100\%$ binary fraction in our COSMIC simulations to reduce computation time and scale the simulations to models which assume a constant $50\%$ binary fraction (model F50) or a metallicity-dependent binary fraction (model FZ) in a post-processing scheme. 

Since we are primarily interested in the effects of a metallicity-dependent binary fraction, we use a single set of assumptions for binary interactions. Our choices follow the \texttt{COSMIC} defaults described in \citet{Breivik2020a} except for the treatment of Roche-lobe overflow (RLO). The stability of RLO mass transfer is determined using critical mass ratios resulting from radius-mass exponents \citep{Webbink1985, Hurley2002}, where the critical mass ratio is defined as the ratio of the donor to accretor mass. We assume critical mass ratios following \citet{Claeys2014} which reduce the standard critical mass ratio assumptions from \citet{Hurley2002} for main sequence (MS) donors by $\sim50\%$ from $3$ to $1.6$ based on the models of \citet{deMink2007} and treat WD accretors separately following the models of \citet{Soberman1997}. We increase the mass loss rate from the donor following Equation 11 of \citet{Claeys2014}. The amount of mass lost during RLO from the donor is limited by the overflow factor of the donor radius to its Roche radius following \citet{Hurley2002}. The amount of mass accepted by the accretor is limited to $10$ times the accretor's mass divided by the accretor's thermal timescale. Finally, for RLO mass loss which becomes unstable and leads to common envelope evolution (CEE) we assume that the donor's binding energy is calculated according to the fits detailed in Appendix B of \citet{Claeys2014} and that orbital energy is deposited with $100\%$ efficiency into unbinding the common envelope ($\alpha=1$).

\subsection{Metallicity-dependent binary fraction}
We fit the results presented in \citet{Moe2019} using linear regression to obtain a piecewise relation between the metallicity, [Fe/H] and binary fraction $f_{\rm{b}}$ as

\[f_{\rm{b}}= \begin{cases} 
      - 0.0648 \cdot [\rm{Fe/H}] + 0.3356, & [\rm{Fe/H}]\leq -1.0 \numberthis \\
      -0.1977 \cdot [\rm{Fe/H}] + 0.2025, & [\rm{Fe/H}]> -1.0  
   \end{cases}
\]


\noindent We convert between [Fe/H] and metallicity Z, assuming all stars have solar abundance such that
\begin{equation}
\label{eq:metallicity}
    [\rm{Fe/H}] = \log_{10}\Big(\frac{Z}{Z_{\odot}}\Big),
\end{equation}
\noindent where we assume $Z_{\odot}=0.02$.

\subsection{A metallicity-dependent SFH: Convolving with the FIRE models}
\label{sec:FIRE}
%\katie{I think this paragraph gives a bit too much detail on FIRE. I think the only detail we need is (1) we use Galaxy m12i from the Latte Suite of the FIRE simulations and (2) Latte reproduces parts of the Milky Way (see Lamberts+2019 paper for this info)}
To create Milky Way-like galaxies which integrate the metallicity-dependent binary fraction, we use the metallicity-dependent ages and positions of galaxy \textbf{m12i} from the ``Latte" suite of the FIRE simulations \citep{Hopkins2015, Wetzel2016, Hopkins2018} to create synthetic, Milky Way-like DWD populations. %\sout{Specifically, the new “FIRE-2” code version was used, which updates the FIRE physics modules in the code GIZMO (Hopkins et al. 2018;  Gabyurov and Nitadori 2011; Hopkins 2015). This update leverages recent developments of mesh-free Godunov hydrodynamics methods, and incorporates updates to “cooling and recombination rates, gravitational force softening, and numerical feedback coupling” (Hopkins et al. 2018). The Latte suite is then run using Gizmo gravity, plus hydrodynamics code in meshless finite-mass (MFM) mode (Hopkins 2015) and the FIRE-2 physics model (Hopkins e al 2018)” $<$- citation}.

The \textbf{m12i} galaxy provides particle mass resolution of 7070 M$_\odot$ per star particle. Each star particle has an associated metallicity, position, and age, which is combined with the output of \cosmic\ to assign DWDs to each star particle by matching its metallicity to our \cosmic\ metallicity grid. The positions of each DWD are assigned using the Ananke framework since multiple DWD binaries can form within a single star particle. Specifically, we use an epanechnikov kernel where the kernel size is inversely proportional to the local density to assign the radial component of spherically symmetric offsets from the center of each star particle following \citet{Sanderson2020}. 

\begin{figure}
	\includegraphics[width=\columnwidth]{figures/sfh_vs_fb.pdf}
    \caption{The metallicity-dependent binary fraction for close solar-type binaries with P $< 10^4$ days (black) plotted against the logarithm of Z/Z$_\odot$, with scatter points denoting the location of the metallicity grid used in our \cosmic\ simulations. The secondary axis shows the amount of mass in stars formed within each metallicity bin from galaxy \textbf{m12i} of the ``Latte" suite in FIRE-2 simulations as a red histogram. The amount of stellar mass formed with super-solar metallicities dominates the distribution. Note that the primary axis shows $f_{\rm{b}}(Z)$ in linear scale, and the secondary axis shows amount of mass formed in log-scale. The opposing trends of these two distributions compete throughout this study.}
    \label{fig:sfh_vs_fb}
\end{figure}

The metallicity-dependent binary fraction is shown in black in Figure \ref{fig:sfh_vs_fb} along with the mass in star particles from galaxy \textbf{m12i}, shown in red, as a function of our metallicity grid. %Plotted in black is the binary fraction as a function of metallicity. 
%The metallicity values chosen for our grid are shown as the scatter points. The red histogram depicts the total mass in stars formed for each metallicity bin within FIRE galaxy \textbf{m12i}, the Milky Way-like simulation used in this study which will be described further in Section \ref{sec:simulations}, as a function of metallicity. 
The binary fraction, $f_{\rm{b}}$, drops drastically across metallicity while the mass formed in \textbf{m12i} increases significantly. These two opposing trends compete throughout this study along with the impact of metallicity on single star evolution to form the final numerical distribution of systems in our DWD populations. 

Since our \cosmic\ simulations assume a binary fraction of $f_{\rm{b}} = 1$, we scale the amount of mass sampled at ZAMS required to produce our \cosmic-generated population of DWDs ($M_{\text{b,ZAMS}}$) to the proper amount of mass sampled in single and binary stars ($M_{\text{ZAMS,sim}}$) for each binary fraction and model. We do this by sampling single stars and primary masses of binary stars from the \cite{Kroupa2001} IMF and sampling secondary masses of the binary stars from a uniform mass distribution, where the number of binaries is determined by the binary fraction model (F50 or FZ). From this sample, we obtain the ratio of mass in single stars to the mass in binary stars as a function of the binary fraction, $R(f_{\rm{b}})$. For model F50, the ratio is a constant $R(f_{\rm{b}}) = 0.64$. For model FZ, the ratio increases from 0.68 at low metallicities to 3.17 at solar metallicity indicating less mass in binaries relative to single stars. The total amount of ZAMS mass in single and binary stars is then $M_{\text{ZAMS,sim}} =  M_{\text{b,ZAMS}} (1 + R(f_{\rm{b}}))$. 

%We use these ratios to then determine the total amount of ZAMS mass sampled in single and binary stars. 
%This ratio can then be used to scale the total amount of ZAMS mass sampled in binary stars required to produce our simulated population from \cosmic\ to $M_{\rm{ZAMS,sim}}$, the total amount of ZAMS mass in single stars and binary systems required to produce the simulated population for a given binary fraction $f_{\rm{b}}$.


%For a given metallicity in our grid and a given DWD type of one compact object binary type, the general procedure to create the Milky Way like galaxy is as follows. We specify metallicity bin $i$ with metallicities Z $\in$ [Z$_{i-1}$, Z$_i$]. The number of DWDs per unit mass N$_{\text{DWD,}i}$ for the \katie{\sout{cosmic-pop} simulated \cosmic} population pop$_i$ with Z = Z$_i$ is found through:

%\begin{align*}
%    R_{m}(f_i) &= \frac{M_{\text{singles,}\odot}(f_i)}{M_\text{binaries,} \odot(f_i)} \numberthis \\
%    \\
%    M_{\text{tot,}i} &=  M_{\text{binaries,}i} \cdot (1 + R_m(f_i)) \numberthis \\
%    \\
%    N_{\text{DWD,}i} &= \frac{N_{\text{DWD, tot,}i}}{M_{\text{tot,}i}} \numberthis
%\end{align*}

%R$_m$ is the fraction of mass in single stars to mass in binary stars for binary fraction $f_i$. \katie{I wonder if we can rewrite these equations to be a bit less confusing; I think we should use Lamberts+2019 as a guide.} \katie{\sout{This is calculated ahead of time using the \cosmic\ independent sampler in the InitialBinaryTable class.} I think that this sentence should say that the scaling incorporates the same assumptions for primary and secondary mass distributions as are used in the simulated data.} One million DWD binaries are initialized at solar metallicity for a given binary fraction, along with the total mass in singles vs. binaries. M$_{{\rm{binaries, }}i}$ the mass in binaries of pop$_i$ (the total mass of the population, as they were run with f$_{\rm{b}}$ = 1.0). M$_{{\rm{binaries, }}i}$ is scaled using R$_m$ to find the total astronomical mass M$_{\rm{tot, Z}_i}$ of the population. N$_{{\rm{DWD, tot,}}i}$ is the total number of DWDs in pop$_i$. 

Once we determine the total ZAMS mass required to produce our simulated population for a given metallicity, the number of DWDs formed per unit solar mass at metallicity $Z_i$ is
\begin{equation}
    n_{\rm{DWD}}(Z_i) = \frac{N_{\rm{DWD, sim}}(Z_i)}{M_{\rm{ZAMS, sim}}(Z_i)}.
\end{equation}

\noindent The number of DWD’s per \textbf{m12i} star particle at metallicity $Z_i$ is then
\begin{equation}
    N_{\rm{DWD},\star} (Z_i) = n_{\rm{DWD}}(Z_i)\, M_\star,
\end{equation}


\noindent where $M_\star = 7070\,M_\odot$ is the mass per \textbf{m12i} star particle. Since $N_{\rm{DWD},\star} (Z_i)$ is not an integer, we treat the decimal component as the probability that the star particle contains an extra DWD in addition to the integer number. For each star particle, we sample with replacement $N_{\rm{DWD},\star}(Z_i)$ DWDs from the corresponding simulated \cosmic\ population at that metallicity and assign the ZAMS birth time of each DWD to the formation time of the star particle. For most DWD types there is more than one DWD binary system assigned to each \textbf{m12i} star particle. 

%Already covered above
%In order to ensure that each DWD position is unique, we assign positions inside the star particle following the Ananke framework which provides offsets from the star particle center. 

%\katie{I think this should go later}This gives us a final distance $d$ to the sun for all the systems, and from this and other information, we can calculate parameters relevant to the LISA observatory like Amplitude Spectral Density (ASD), signal-to-noise ratio (SNR), chirps, chirp masses, etc, as discussed in the next section.

%This will be a floating point number. The integer component gives one number of DWDs to sample with replacement from the cosmic-pop output to apply to each FIRE star particle. The decimal component will be treated as a probability $p$ of a binary existing for that particle. We assign an additional DWD to each star particle with probability $p$. The DWDs now have metallicities, ages, and galactic positions. 

%\katie{I think the presentation of the next couple of paragraphs is a bit out of order. What about presenting it in the order: (1) we evolve the DWDs from their formation time up to the present day using the Peters equations (2) we discard any systems which overflow their roche lobes.}

If the DWD formation time is less than the age of the star particle, we evolve the DWD over the remaining time between its formation and star particle age, $t_{\rm{evol}}$, to produce the present-day population. Once a DWD is formed, we assume that the binary evolves only due to the emission of GWs. Due to tidal effects and mass transfer between their progenitor binaries, all DWDs in our simulations are circular, thus eccentricity does not need to be considered. The orbital evolution over the time $t_{\rm{evol}}$ is then simply defined according to \citet{Peters1964} as
\begin{equation}
    a_f = (a_i - 4\beta t_\text{evol})^{1/4},
\end{equation}
where $a_i$ is the DWD separation at formation, and 
\begin{equation}
    \beta = \frac{64G^3}{5c^5} M_1M_2(M_1+M_2)
\end{equation}
is constant throughout DWD evolution \citep{Peters1964}.

We discard any DWDs for which the sum of their ZAMS birth time, given by the star particle formation time, and DWD formation time is larger than the age of the star particle since the system will not have evolved long enough to become a DWD at present. We further discard any DWDs for which the lower mass WD overflows it's Roche lobe before present day, because the outcomes of these interactions are highly uncertain and their treatment is outside the scope of this work \citep[e.g., ][]{Shen2015, Kremer2017}. The separation at which the lower mass WD overflows its Roche Lobe is defined as

\begin{equation}
    a_{\text{RLO,}\ell} = R_{\ell} \frac{0.6 q_{\ell}^{2/3} + \ln{(1+q_{\ell}^{1/3})}}{0.49 q_{\ell}^{2/3}}
\end{equation}
where $R_{\ell}$ is the radius of the lower mass WD and $q_{\ell} = M_{\ell}/M_{h}$ is the ratio of the lower- to higher-mass WD components. \citep{Eggleton1983}. 
We define the radius of a WD following \citet{Tout1997, Hurley2000} as
\begin{equation}
    R_{\rm{WD}} = \max\Bigg(R_\text{NS}, 0.0115\sqrt{\left(\frac{M_\text{Ch}}{M}\right)^{2/3}-\left(\frac{M}{M_\text{Ch}}\right)^{2/3}} \Bigg) 
\end{equation}
\noindent where $R_\text{NS} = 1.4\cdot 10^{-5}$ R$_\odot$ is the radius of a neutron star, $M_{\text{Ch}}=1.44$ M$_\odot$ is the Chandrasekhar limit for the mass of a stable WD, and $M$ is the mass of the DWD in solar masses. 

For the non-discarded systems, we log the present-day separations from which the present-day orbital frequency $f_\text{orb}$ can be found using Kepler's third law. The GW frequency is then $f_\text{GW} = 2f_\text{orb}$.

%Following the creation of an initial population, certain systems are discarded. We discard any systems with formation times longer than their assigned age from their respective star particle since these systems won’t have formed a DWD by present day and would thus not be observable by LISA. The time from DWD formation until the time when the system merges, t$_{\text{merge}}$, is defined as 
%\begin{equation}
%    t_\text{merge} = \frac{a_i^4}{4 \beta}
%\end{equation}
%where $a_i$ is the DWD separation at formation, and 
%\begin{equation}
%    \beta = \frac{64G^3}{5c^5} M_1M_2(M_1+M_2)
%\end{equation}
%is constant throughout DWD evolution \citep{Peters1964}. The delay time of the %system, $t_\text{delay}$, is defined as the sum of the DWD formation time and t$_{\text{merge}}$. Systems are also discarded if the delay time is less than the FIRE age, since they would have merged before present day. Finally, we 
%We find the time corresponding to when} the system will reach separation $a = a_\text{RLOF,2}$ by inverting Eq.(12), and \textcolor{purple}{\sout{discarding systems with $t_\text{phys} + t_\text{RLOF} \leq t_\text{FIRE}$} discard systems that would overflow their Roche Lobe before present day according to their FIRE age}. 

%For the non-discarded systems, we assume evolution occurs solely through the emission of gravitational radiation. Furthermore, 


%We discard any sampled DWDs for which the sum of their ZAMS birth time given by the star particle and DWD formation time is larger than the age of the star particle. 
%since DWDs orbit mainly in the second harmonic, with negligible signal stemming from the others. Systems in the LISA band are defined to have $f_\text{GW} \in [10^{-4}, 1.0]$ Hz, although they may not be observable from Earth with LISA’s current sensitivity limits.


%\katie{I think we might want to leave photometric observations to a future paper}
%In order to investigate possible follow-up with photometric observations, we also calculate the bolometric magnitude of the DWD. To calculate the luminosity of the DWD, we utilize the modified Mestel cooling described in Hurley(2003):
%\begin{equation}
%    L_{\text{WD,}i} = %\frac{bM_iZ^{0.4}}{(A_i(t_\text{evol,i}+0.1))^x}
%\end{equation}
%where $M_i$ is the WD mass in solar masses and $Z$ is the metallicity of our specified metallicity bin. $A_i$ is the baryon number for the WD subtype. 4 corresponds to helium, 15 to carbon-oxygen and 17 to oxygen-neon. $t_\text{evol, i}$ is evolution time for WD component $i$ in Myr (i.e. $t_{\text{evol, 2}}$ = $t_{\text{evol}}$, but $t_{\text{evol,1}} = t_{\text{FIRE}}-t_{\text{phys,1}}$ where $t_{\text{phys,1}}$ must be retrieved from \cosmic\ at primary formation.). The values of $b$ and $x$ are dependent on this age:
%\begin{equation}
%x = \begin{cases} 
%     1.18 & t_\text{evol, i} < 9000.0\\
%      6.48 & t_\text{evol, i} \geq 9000.0
%   \end{cases}
%\end{equation}
%\begin{equation}
%b = \begin{cases} 
%      300 & t_\text{evol, i} < 9000.0\\
%      300(9000A)^{5/3} & t_\text{evol, i} \geq 9000.0 
%   \end{cases}
%\end{equation}
%We can then find the bolometric magnitude of the system using https://www.astro.princeton.edu/~gk/A403/
%constants.pdf and https://www.astro.keele.ac.uk/jkt/pubs/
%JKTeq-fluxsum.pdf:
%\begin{equation}
%    m_{\text{bol, tot}} = -2.5 \cdot \log_{10}(10^{-0.4m_1}+10^{-0.4m_2}),
%\end{equation}
%where the bolometric magnitude of each component $m_{i}$ is %given by:
%\begin{equation}
%    m_{i} = 4.8 - 2.5 \cdot %\log_{10}\left(\frac{L_i}{L_\odot}\right)
%\end{equation}


\section{LISA detectability}
\label{sec:LISA_obs}
We use \legwork\footnote{https://legwork.readthedocs.io} \citep{Wagg2021} to determine the detectability of our simulated DWD populations for sources with GW frequencies $f_{\rm{GW}}>10^{-4}\,\rm{Hz}$. \legwork\ calculates the position-, orientation-, and angle-averaged signal to noise ratio (SNR) for inspiraling GW sources closely following the derivations of \citet{Flanagan1998} and using the LISA noise power spectrum density (PSD) of \citet{Robson2019}. 

To lowest order in the post-Newtonian expansion, the frequency evolution of circular orbits for quadrupole GW emission is defined as
\begin{equation}
    \Dot{f}_n = \frac{48n}{5\pi} \frac{(G\mathcal{M}_c)^{5/3}}{c^5} (2\pi f_\text{orb})^{11/3}.
\end{equation}
We classify DWDs as evolving, or ``chirping", when $\Dot{f}_n \geq 1/t_\text{obs}^2$. For evolving sources, the SNR is

\begin{equation}
   \langle \rho \rangle^2_{\text{circ,evol}} = \int_{f_0}^{f_1}df   \frac{h_{c}^2}{f^2S_n(f)}
\end{equation}

\noindent where $h_c$ is the characteristic strain of the system, $S_n(f)$ is the LISA sensitivity curve of \citet{Robson2019}, and the frequency limits are determined by the orbital evolution over the observation time, $T_{\rm{obs}}=4\,\rm{yr}$. The characteristic strain for %DWDs 
circular orbits is 
%only defined for the $n=2$ harmonic and is:
\begin{equation}
    h_{c}^2 = \frac{2^{2/3}}{3 \pi^{4/3}} \frac{(G \mathcal{M}_c)^{5/3}}{c^3 D_L^2} \frac{1}{f_{\rm orb}^{1/3}},
\end{equation}
\noindent where $\mathcal{M}_c = (M_1M_2)^{3/5}/(M_1 + M_2)^{1/5}$ is the system's chirp mass, and $D_L$ is the systems luminosity distance which we assume to be the distance of each simulated DWD to the Sun. 

For stationary sources, the SNR is modified due to the lack of observable orbital evolution as
\begin{equation}
    \rho_{\text{circ,stat}}^2 = \frac{h_{2}^2T_{\text{obs}}}{4 S_n(f_2)}
\end{equation}
with the observation time $T_{\text{obs}}$. Here, $h_2$ is strain amplitude of the source for the second orbital frequency harmonic
\begin{equation}
   h_2^2 = \frac{2^{22/3}}{5}\frac{(G\mathcal{M}_c)^{10/3}}{c^8D_L^2}(\pi f_\text{orb})^{4/3}.
\end{equation}
and is connected to the characteristic strain as
\begin{equation}
    h_2^2 = \frac{\dot{f}_{2}}{f_{\rm{orb}}^2} h_c^2.
\end{equation}

%$g(n,e)$ is the relative power of gravitational radiation at the nth harmonic taken from Peters and Mathews(1963). For $n=2$ this reduces to:
%\begin{equation}
%    h_2^2 = \frac{2^{22/3}}{5}\frac{(G\mathcal{M}_c)^{10/3}}{c^8D_L^2}(\pi f_\text{orb})^{4/3}
%\end{equation}
The amplitude spectral density for a stationary system is finally defined as $ASD = h_{2} \sqrt{T_\text{obs}}$, such that the SNR for stationary source is simply, $\rho \sim ASD/S_n$.

The Galactic foreground included in the \citet{Robson2019} LISA noise curve was generated using a different binary evolution code and set of model assumptions for DWD formation and evolution \citep{Toonen2012, Korol2017}. Thus, we use the detector curve only and generate an approximate foreground from each of our populations as follows. Instead of performing a full source subtraction algorithm \citep[e.g.][]{Littenberg2020}, which is out of the scope of this work, we calculate the PSD of the Galactic DWD population with a frequency resolution set by the LISA mission time as $1/T_{\rm{obs}}\sim1/4\,\rm{yr}^{-1}\sim8\times10^{-9}\,\rm{Hz}$. We then approximate the foreground as the running median of the PSD with a boxcar window with a width of $10^3$ frequency bins similar to \citet{Benacquista2006}. The Galactic DWD PSD is truncated near $10\,\rm{mHz}$ for both of our models because we remove all DWDs which experience Roche-lobe overflow. In order to smooth the effect of this truncation in our foreground, we fit each running median with fourth-order polynomials for GW frequencies up to $1\,\rm{mHz}$, thus allowing an approximation of the foreground PSD for higher frequencies. These fits are listed in Table~\ref{tbl:fits} where the polynomial is described as
\begin{equation}
\label{eq:fit}
    \log_{10}(\rm{confusion\ fit}/\rm{Hz}) = a\,x^4 + b\,x^3 + c\,x^2 + d\,x + e
\end{equation}
\noindent and $x=\log_{10}(f_{\rm{GW}}/\rm{Hz})$. We add the fitted polynomial of the PSD's running median to the LISA noise PSD to obtain a sensitivity curve for each model. 

\begin{table}[]
    \centering
    \begin{tabular}{|c|c|c|c|c|c|}
       \hline
       model & a & b & c & d & e \\
       \hline
       \hline
       F50 & -223.5 & -189.8 & -76.8 & -14.0 & -1.0 \\
       \hline
       FZ & -558.5 & -575.4 & -243.1 & -45.8 & -3.2 \\
       \hline
    \end{tabular}
    \caption{Polynomial fitting coefficients for the confusion foreground fit of Equation~\ref{eq:fit} for each binary fraction model.}
    \label{tbl:fits}
\end{table}

\section{Metallicity effects on the formation and evolution of DWDs}\label{sec:results}

\subsection{DWD types and their formation channels}\label{sec:ini}
As discussed in Section\,\ref{sec:bin_pop}, we consider four DWD sub-types, which each contribute differently to LISA's GW signals: He + He, CO + He, CO + CO, and ONe + X. Each sub-type has a unique distribution in their formation times, initial masses, radii, and orbital periods stemming from variations in their evolution channels and their formation efficiency. Here we describe the general formation scenarios and population properties of Galactic close DWDs which may be observable by LISA.
%We thus simulated \cosmic \ DWD populations separately for each DWD type and metallicity value in order to identify variations in how each type is impacted differently by changes in metallicity. A key effect is that in general, lower metallicity stars have shorter MS lifetimes and smaller initial radii than high metallicity counterparts (https://arxiv.org/pdf/2001.10404.pdf). This is discussed in greater depth later in this section. 


%The resulting DWD populations from \cosmic\ gives insight in to the impact that metallicity has on these systems upon their initial formation. The trends we see within these distributions can be better interpreted through the characteristics and evolutionary channels of each DWD type which are outlined below. 

%He + He: (REF = https://iopscience.iop.org/article/
%10.1086/303686/fulltext/34537.text.html)
He WDs are unable to form through single star evolution within the lifetime of the Milky Way. Instead, they originate through interactions in close binary systems or binaries with large eccentricities. Because of this, He WDs are able to form with low component masses on order $\sim$ 0.1M$_\odot$, with the majority of He WDs in our simulations having masses between 0.2-0.5 M$_\odot$. 
%, which would be impossible through SSE within a Hubble time (REF). 
He + He DWDs form through evolution of close binary systems, during which their envelopes are both stripped through RLO and CE phase interactions before Helium ignition occurs. The two progenitor stars generally have masses $\lesssim 3 M_\odot$ which is lower than the progenitors of other DWD types. Our simulated He + He DWDs have an approximately constant distribution of formation times $\gtrsim$ 2.5 Gyr. Lastly, since the ZAMS separations are skewed towards shorter values, we also see that the resulting DWD separations are smaller on average than that of other DWD types. 

\begin{figure*}
	\includegraphics[width=\textwidth]{figures/form_eff.pdf}
    \caption{The DWD formation efficiency vs metallicity of DWD populations simulated with \cosmic. Each panel shows the formation efficiency for a given DWD type. The solid lines indicate the formation efficiency for model FZ which incorporates a metallicity-dependent binary fraction. The dashed lines indicate the formation efficiency for model F50, which assumes a constant binary fraction of $50\%$. The DWD formation efficiency drops by a factor of $4-5$ for model FZ and a factor of 2-5 for model F50. See Section~\ref{sec:formeff} for a careful description of the trends for each DWD type.}
    \label{fig:form_eff}
\end{figure*}


%CO + He: 
%(REF = https://www.researchgate.net/publ ication/231115182\_Helium\_and\_Carbon-Oxygen\_White\_ Dwarfs\_in\_Close\_Binaries)
A CO WD forms when a star is able to begin the helium burning process before its envelope is stripped. Thus to form a CO + He DWD, RLO and CE stages occur after one component experiences core helium burning, but before the other component can. Most close CO + He DWDs form in approximately 2 Gyr after ZAMS and with very short periods because the He WD is formed through the ejection of a common envelope which greatly reduces the orbital separation. Because of these short formation separations, many CO + He DWDs merge before the present day.
%(Lamberts) of $O$(hour), and may merge due to GW radiation by present day. Specifically, DWDs with sub-hour periods tend to merge in less than 10$^8$ years (REF). 
Due to their asymmetric mass distributions, they have lower chirp masses, but their shorter periods make them important candidates for LISA detection. 
%Possible formation channels of these short period CO + He DWDs are explored in (REF). The fate of the binary is dependent on the orbital period that results from the final CE phase before DWD formation. 

%CO + CO:
%A CO WD forms when a star is able to begin helium burning before its envelope is stripped (Lamberts,  https://arxiv.org/pdf/2011.10439.pdf). 
To prevent the two stars' envelopes from being stripped before helium ignition, CO + CO DWDs typically form from progenitors in wider orbits, and the two components may have little to no interaction during their evolution from ZAMS to DWD. CO DWDs thus have a distribution in progenitor separation that extends to larger values than for other DWD types. Most need $> 0.3$ Gyr to form. These systems have component WD masses between $0.35$--$1.0\,M_\odot$ and make up the majority of the DWD population. 




%ONe + X: (REF = https://arxiv.org/pdf/2011.10439.pdf)
ONe WDs are rare and typically form from massive progenitor stars which evolve through the asymptotic giant branch (AGB) phase, thus resulting in a higher mass WD. All ONe WDs in our \cosmic\ populations, e.g., have progenitor ZAMS masses above $4\,M_\odot$, and the resulting ONe WDs have a relatively flat distribution of masses from $1.05\,M_\odot$ up to the Chandrasekkhar limit of $1.4\,M_\odot$. Because an ONe WD can have a companion of any other WD type in our study, there is a spread in their distributions for separation, secondary mass, final orbital period, and formation time. In general, however, these systems result from wider separations to allow for the evolution of the ONe component without merging, e.g. all initial separations in our \cosmic \ populations have separations $\gtrsim 1.5\,R_\odot$. ONe + X DWDs can form on short timescales, as low as $30\,\rm{Myr}$ for the majority of high-metallicity systems. 



\subsection{Metallicity-dependent trends in the formation efficiency of DWDs}\label{sec:formeff}
The number of DWDs formed per unit solar mass of ZAMS star formation, or DWD formation efficiency $\eta_{\rm{form}}(Z))$, varies with metallicity. Consequently, a metallicity-dependent binary fraction further impacts the efficiency of DWD formation within the Galaxy. Figure \ref{fig:form_eff} shows the DWD formation efficiency as a function of metallicity for each DWD type and for each of our binary fraction models. In general, the formation efficiency decreases with increasing metallicity. This effect is exaggerated for model FZ which assumes a binary fraction which also decreases with increasing metallicity. 

%KB cut 28 sept 21
%This drop across metallicity is due to an increase in the number of merging systems before the formation of the DWD.
%The increased propensity for merging during binary interactions stems from variations in the star's intrinsic properties caused by the effect of metallicity on stellar evolution. A higher abundance of metals results in a cooler, dimmer star with a lower central temperature and that is less compact, leading to larger stellar radii and longer main sequence lifetimes. These changes further alter the timescales for Roche-lobe overflow interactions.

%Consider two stars of high and low metallicity but equal mass. The star with higher metal abundance will have a larger radius, and a higher fraction of mass in the convective envelope (which creates longer turnover timescales) of the star/less in the core. This in theory impacts the CE phase - there would be more envelope to eject, leading to a longer energy transfer times, longer CE phases, and thus more mergers. We will now detail the unique processes for each DWD type which leads to the drop in DWD formation efficiency. For many of the subtypes, the mechanisms behind and degree to which the DWD formation efficiency is impacted can be identified within subsets of their initial orbital period.
%\katie{I think we might want to shorten this previous paragraph a bit since we go into detail later on.}


For He + He DWDs, the sudden drop in formation efficiency near $\log(Z/Z_{\odot})=-1.0$, is generally caused by the timescale for which the initially more massive star in the DWD progenitor overflows it's Roche lobe. At lower metallicities, donors tend to fill their Roche lobes while they are still on the main sequence and the mass transfer remains stable. This is because, in our \cosmic\ models %\snotes{[in just our \cosmic\ models or astrophysically? or are we stating this is just for \cosmic\ to be safe?]}, 
% KB: yeah exactly -- to be safe
lower metallicity stars evolve faster than high-metallicity stars and have larger radii near the end of the MS. At higher metallicities, mass transfer is initiated when the donor has left the main sequence and the binary enters a common envelope evolution. The ZAMS orbital period leads to different specific evolutionary channels. For short-period systems with periods below $10~\rm{days}$, this leads to a stellar merger due to insufficient orbital energy to eject the envelope. Stellar mergers continue to dominate the evolutionary pathways of systems with intermediate orbital periods ($1 < \log_{10}(P_{\rm{orb}}/\rm{day}) < 2.5$). However, an additional growing number of merging systems arises with one WD component and one stellar companion, and systems which don't interact at all and thus do not form a He + He DWD. The distinction between the various scenarios in this intermediate orbital period range depends on the combination of their ZAMS masses and orbital periods. Finally, at wider initial periods $\log_{10}(\rm{P}) > 2.5$, the decrease in formation efficiency is dominated by systems which never interact and thus do not form a DWD before the present day.

\begin{figure*}
	\includegraphics[width=\textwidth]{figures/CEsep.pdf}
    \caption{Average interaction separation, $\overline{\mathrm{a}}_{\rm{CE}}$ of progenitors of close DWDs across metallicity for each DWD type from model FZ. Solid lines show the average separation at the first RLO for binaries in each metallicity bin. The shaded regions show the $1\sigma$ spread around the mean within each metallicity bin. The average interaction separation increases with metallicity for every DWD type. The positive trend in the average interaction separation is a direct consequence of larger envelope masses of higher-metallicity donors which are less evolved than their lower-metallicity counterparts.} 
    %The initial radii of high-metallicity ZAMS stars is larger than their lower-metallicity counterparts but inflate less during their evolution, creating three cases: 1) stars with wider initial separations that do not shrink enough to evolve into the LISA band before present day or fail to interact at all since their radii never increase enough to interact, 2) systems which remain stable throughout their evolution and end up in the LISA band, which requires higher interaction separations to account for the larger radii and deeper convective envelopes which need to be ejected (i.e the separation must be large enough to allow for longer CE phases), and 3) systems which merge before present day either through interactions delayed until late in their evolution from radii which do not evolve much, causing unstable mass transfer, or through very close initial separations which merged during a CE phase before present. Only case (2) would result in successful systems shown here.}
    \label{fig:CEsep}
\end{figure*}


For short-period CO + He DWD progenitor binaries with orbital periods below $\sim30~\rm{days}$, the only channel for mergers before DWD formation is during a CE evolution. This channel is similar to the stellar merger channel for He + He DWDs, but due to higher-mass progenitors the CE evolution results in the binary component with a higher mass becoming a CO WD. At higher metallicities, the CO WD merges with its companion. Higher-metallicity progenitors have a larger fraction of their mass in the convective envelope when compared to lower-metallicity stars of the same mass \citep{Amard2019, Amard2020} Thus a CE phase with a higher-metallicity progenitor requires more orbital energy to eject the common envelope, causing a larger amount of orbital shrinking which results in a merger later in its binary evolution.

For CO + He DWD progenitor systems with intermediate orbital periods ($1.5\leq \log_{10}(P_{\rm{orb}}/\rm{day})< 2.5$) the mechanisms which impact formation efficiency are complex. The dominant way to lose CO + He DWDs is stellar mergers that occur during the second CE phase. For lower-metallicity systems, the second CE is successful, while the opposite is true for higher-metallicity systems. This is again because the envelope mass of the CE evolution donor at higher metallicity is larger than it would be at lower metallicity and thus requires more orbital energy to eject in the CE phase. This leads to shorter post-CE orbital periods after the first CE phase and hence mergers during the second phase.

If a lower-metallicity system forms a CO + He DWD and a higher-metallicity system does not, this is because the initially more massive binary component initiates a CE while on the giant branch (GB) instead of the AGB. This instead leaves behind a He WD with a stellar companion, which is the dominating scenario that restricts formation efficiency. There are a few edge cases where either a He + He DWD is formed instead, or when a CO WD and a stellar companion is formed and there has been very nearly but not quite enough time for a CO + He DWD to form. 

Long-period binaries with $\log_{10}(P_{\rm{orb}}/\rm{day})>2.5$ also display complex scenarios that hinder CO + He DWD formation. In near-equal contributions, our \cosmic\ simulations produce either stellar mergers or stable non-DWD binaries at the end of the Hubble time. At these orbital periods, stellar mergers always occur with a CE phase between a stellar companion and a CO or He WD. Similar to binaries with shorter orbital periods, the mergers occur because of increased CE donor envelope masses at higher metallicities. A subdominant channel of stable He + He DWDs can also occur when a high-metallicity primary overflows it's Roche lobe while still on the GB and thus forms a He WD.

The decrease in the CO + CO DWD formation efficiency with increasing metallicity stems from different evolutionary channels which arise at the ZAMS orbital period boundary of $\log_{10}(P_{\rm{orb}}/\rm{day})\simeq3$. We find that for binaries with orbital periods below this boundary, the most common way that CO + CO DWDs form at lower metallicities but not at higher metallicities is through stellar mergers during a CE phase with a donor that is still on the GB. For lower-metallicity binaries, which evolve on faster timescales, the primaries enter CE evolution while on the AGB instead and the binary is able to survive. For binaries with orbital periods above the boundary, the vast majority of systems with wide initial orbits end up as stable binaries. The systems which don’t form a CO DWD at high metallicity do so because one or both of the binary components initiate a CE phase while still on the GB thus producing a CO + He or He + He DWD. 

The strongest effect which hinders formations of higher-metallicity ONe + X DWDs is the strength of metallicity-dependent stellar winds assumed in our model \citep{Vink2001}. The strength of line-driven winds varies more strongly for the more massive ($\geq5\,\rm{M}_\odot$) progenitors of ONe WDs relative to the other lower mass WD progenitors. At higher metallicities, ONe WD progenitors can lose enough mass through winds such that they don't ignite their CO cores and thus leave behind a CO WD. Conversely, the lower-metallicity progenitors retain enough mass to cause carbon ignition and leave behind an ONe WD.

\subsection{Metallicity trends in DWD progenitor common envelope separation}\label{sec:CEsep}

\begin{figure*}
	\includegraphics[width=\textwidth]{figures/lisa_nums.pdf}
    \caption{The number of LISA-band systems formed for each DWD type as a function of the base-10 logarithm of metallicity, normalized to solar value. The solid line shows the FZ population with a metallicity-dependent binary fraction incorporated, and the dashed line shows the F50 population for a standard binary fraction of 0.5. The LISA population of DWDs is dominated by stars with super-solar metallicities. This is true even for model FZ, which drops off significantly for higher metallicities, because of the large number of stars formed in \textbf{m12i} beyond Z$\simeq$Z$_\odot$. There is a double peak in the He + He population; the first peak is caused by the sharp drop in formation efficiency past Z$\simeq$0.1$Z_\odot$ which is then greatly overcompensated for by the amount of star formation at higher metallicities which forms the second peak.}
    \label{fig:lisa_nums}
\end{figure*}

All systems that end up radiating GWs in the LISA band have undergone at least one phase of CE evolution. For systems which experience a stable RLO mass transfer in the first interaction, the CE phase plays a key role in shrinking systems with initially wide separations to bring them into the LISA band. 

Figure \ref{fig:CEsep} shows the average separation at the first instance of CE evolution, $a_{\text{CE}}$ of all DWD progenitors that result in systems orbiting in the LISA frequency band at present day, as a function of metallicity for each DWD type. The solid lines denote the average value, and the $1\sigma$ variance is shown in the surrounding shading. The average CE separation increases in general across metallicity. Higher-metallicity binaries will interact earlier in the binary's lifetime than a lower-metallicity binary of equal separation due to their relatively larger maximum radii. Since higher-metallicity binaries are also more likely to merge during CE interactions because of their relatively more massive donor envelopes, the DWDs which survive and eventually orbit in the LISA band originate from systems with higher interaction separations which allow their orbits to shrink significantly during CE phases without merging. This regulation plays a key role in smearing out any observable effects of a metallicity-dependent binary fraction in the population of DWDs observable by LISA. 


\section{Metallicity Dependence of the LISA DWD Population}
\label{sec:LISA_met}

While metallicity impacts the intrinsic properties of our simulated DWD populations as described in Sections~\ref{sec:formeff} and \ref{sec:CEsep}, when we consider the present-day Galactic close DWDs we find that the population detectable by LISA changes only numerically. The number of DWDs in the LISA frequency band decreases by $\sim50\%$ when comparing model F50 to model FZ.


\begin{figure}
	\includegraphics[width=1\columnwidth]{figures/PSD.pdf}
    \caption{The PSD of the entire Galactic DWD GW foreground, summed over all metallicities and DWD types vs GW frequency for models FZ and F50. The vertical lines show the PSD where model F50 is shown in dark blue and model FZ is shown in light blue. The dashed lines show fits to the rolling boxcar median of width 1000 bins for each PSD (see \ref{sec:LISA_obs} for a discussion). A metallicity-dependent binary fraction (model FZ) yields fewer DWDs across all frequencies than a $50\%$ binary fraction (model F50) by roughly a factor of $2$. This produces a lower GW confusion foreground for frequencies with f$_{\rm{GW}}\leq\sim10^{-3}$ Hz.}
    \label{fig:PSD}
\end{figure}


\begin{figure*}
	\includegraphics[width=\textwidth]{figures/LISA_SNR_FZ.pdf}
	\includegraphics[width=\textwidth]{figures/LISA_SNR_F50.pdf}
    \caption{The ASD vs GW frequency for DWDs resolved with SNR $> 7$ for each DWD type where the top row shows the population from model FZ and the bottom row shows the population from model F50. In each panel, the LISA sensitivity curve, including the confusion foreground for each model, is shown in black and the total population for each model is shown in grey. We find that each model qualitatively exhibits similar characteristics and that the only change is in the yield of resolved DWDs for each type based on the strength of the confusion foreground.}
    \label{fig:LISA_SNR}
\end{figure*}

Figure \ref{fig:lisa_nums} shows the number of DWDs orbiting with frequencies $f_{\rm{GW}} > 0.1\,\rm{mHz}$ against metallicity for each DWD type. The solid lines show DWDs from model FZ, and the dashed lines show DWDs from model F50. The He + CO, CO + CO, and ONe + X populations each have strong peaks in the number of DWDs near solar metallicity at which the majority of star formation in galaxy {\bf{m12i}} occurs. The largest contribution to the population comes from metallicities above $\sim$ 0.01Z$_\odot$. The discrepancy between the two binary fraction models is also the most significant above this threshold. When creating our DWD populations the DWD formation efficiency, number of \textbf{m12i} star particles, and binary fraction all compete. The amount of stars formed in \textbf{m12i} at higher metallicity values overwhelms the drop in DWD formation efficiency by multiple orders of magnitude, so this effect dominates when determining the number of stars initially sampled in the population for $f_{\rm{b}}$. 

There are two peaks in the distribution of He + He DWDs. This occurs because near $Z\simeq 0.1Z_\odot$, the DWD formation efficiency transitions from near constant values to a sharp decrease (see Figure \ref{fig:form_eff}). However, for a drop in the formation efficiency by a factor of of $\sim$ six, the amount of star formation in galaxy \textbf{m12i} increases by more than an order of magnitude for supersolar metallicities. Above $Z=Z_\odot$ this overcompensates for the efficiency drop, producing the second peak.


\begin{figure*}
	\includegraphics[width=\textwidth]{figures/Mc_vs_dist.pdf}
    \caption{The chirp mass vs distance for each DWD type is shown. Only systems with observable evolution in their GW frequency, i.e those which are chirping, and with SNR $>7$, are shown, since these are systems for which distance can be separated from chirp mass within their strain amplitude. Each panel shows one DWD type, summed over all metallicities. Model FZ is indicated with solid light blue contours, and model F50 is indicated with dark blue dashed contours respectively. Contours are shown at the $5^{\rm{th}}$, 25$^{\rm{th}}$, 50$^{\rm{th}}$, 75$^{\rm{th}}$, and 95$^{\rm{th}}$ percentiles. Despite intrinsic changes to population properties induced by a metallicity-dependent binary fraction, and a reduction in the height of the DWD Galactic foreground the distributions are very similar.}
    \label{fig:Mc_vs_dist}
\end{figure*}


The reduced number of DWDs in model FZ relative to model F50 is also apparent in the GW power spectral density (PSD) LISA will observe. We show the GW PSD of each model, as well as the confusion estimate, in Figure~\ref{fig:PSD}. While model F50 (dark blue) and model FZ (light blue) produce several thousand large spikes in the PSD across LISA's frequency band, the overall foreground height, including the confusion, is larger for model F50. This is a direct consequence of the overall reduction in the size of the close DWD population in model FZ. 

Similar to previous studies, we find that LISA will be able to resolve several thousand DWDs. Figure \ref{fig:LISA_SNR} shows the amplitude spectral density vs GW frequency of the resolved systems with SNR $> 7$ for each DWD type, where the top and bottom rows show results for models FZ and F50 respectively. For comparison, we also show the LISA sensitivity curve, including the 
modeled confusion foreground from each population in black, and the the entire population for each model in grey. Apart from each DWD type having a different abundance of resolved systems, the population-wide characteristics remain unchanged between the two binary fraction models. The populations containing at least one He WD occupy the lower-ASD, higher-GW frequency region of parameter space compared to the total population, with CO + He DWDs having larger ASDs than the He + He DWD population. Conversely, DWD types without a He WD component tend to occupy the higher-ASD, lower-GW frequency region of parameter space. This difference is largely due to the formation scenario of DWDs containing a He WD, which form from the ejection of a common envelope created by the He WD progenitor. These lower mass progenitors overflow their Roche lobes at closer separations relative to the higher mass progenitors (e.g. Figure~\ref{fig:CEsep}) and thus also produce closer DWDs. While the distance to any one DWD strongly influences its ASD, DWD populations without a He WD component have, on average, higher ASDs due to their more massive WD components.

We note that while the height of the confusion foreground and the number of DWDs radiating GWs with $f_{\rm{GW}} > 10^{-4}\,\rm{Hz}$ is reduced by a factor of 2 for model FZ relative to model F50, the number of resolved sources is not reduced to an equal degree. This is because in the absence of less competing GW signals from DWDs in the LISA frequency band, more DWDs can be individually resolved. Between models FZ and F50, the number of resolvable systems with SNR $>7$ over all DWD types decreases by only $\sim14\%$. 

The distance and chirp mass of DWDs which exhibit observable orbital evolution due to the emission of GWs during the LISA mission can be measured. This is because the chirp mass -- distance degeneracy in the observed strain can be broken with the observed GW frequency evolution, or chirp. Assuming a chirp resolution of $1/T_{\rm{obs}}^2 \sim 8\times10^{-9}\,\rm{Hz}^2$, we select the DWDs whose chirp masses and distances can be measured. Figure \ref{fig:Mc_vs_dist} shows the chirp mass vs the luminosity distance for each DWD type in this selected population. The contours show the $5^{\rm{th}}$, $25^{\rm{th}}$, $50^{\rm{th}}$, $75^{\rm{th}}$, and $95^{\rm{th}}$ percentiles for models FZ (light blue) and F50 (dark blue, dashed). Despite the reduction in the height of the confusion foreground when considering model FZ relative to F50, we find that LISA is unable to differentiate between the chirp mass -- distance distributions of the two models.

\section{Binary evolution assumption variations}
\label{sec:model_compare}

\begin{figure}
	\includegraphics[width=0.45\textwidth]{figures/model_comp.pdf}
    \caption{The top panel shows the Galactic DWD confusion fit vs GW frequency for different binary evolution assumption variations (colors) with model F50 shown in dashed lines and model FZ shown in solid lines. The bottom panel shows the ratio of the number of DWDs orbiting in the LISA frequency band for model FZ to model F50 vs to number of DWDs orbiting in the LISA band for model F50 only. While both the height of the confusion foreground and number of LISA DWDs changes for each variation, the FZ models within each variation exhibit a constant reduction by a factor of two compared with the F50 models.}
    \label{fig:model_comp}
\end{figure}

In order to test the robustness of the reduction of the height of the Galactic DWD GW foreground when assuming a metallicity-dependent binary fraction, we repeat our analysis using three different binary evolution assumption variations. For each variation, we consider models FZ and F50 as done in the fiducial case described above. In variation $q3$, we vary the assumption for the critical mass ratios at which a RLO interaction remains stable or becomes unstable from our fiducial assumptions such that the critical mass ratio is increased to $3.0$ and thus allows stable mass transfer for more massive RLO donors. In variations $\alpha25$ and $\alpha5$, we modify the common envelope ejection efficiency to be either much less ($\alpha=0.25$) or more ($\alpha=5$) than in our fiducial assumption ($\alpha=1$). Larger common envelope ejection efficiencies lead to wider post-CE separations, while smaller ejection efficiencies either lead to closer post-CE separations or mergers where the envelope ejection fails. 

In each variation, changing the binary evolution assumptions dramatically changes the formation and evolution of the DWD populations. These changes lead to large shifts in the overall number of close DWDs in our synthetic present-day Milky-Way-like galaxies. We find that the total number of DWDs with $f_{\rm{GW}}>10^{-4}\,\rm{Hz}$ increases for both variations $q3$ and $\alpha5$. This is because there are fewer stellar mergers which occur before the formation of a DWD, thus allowing more systems to evolve due to GW emission and orbit in the LISA frequency band at present. Conversely, for variation $\alpha0.25$, we find that the number of DWDs orbiting with frequencies in the LISA band is drastically reduced. This is because of the highly inefficient use of orbital energy to eject the common envelope, thus producing more stellar mergers, or closer binaries which are more prone to future mergers.

Interestingly, when we compare the populations of each binary fraction model for our variations, we find that the number of close DWDs reduces by the constant factor of two as seen in our fiducial set of assumptions. This is illustrated in \ref{fig:model_comp}. The top panel shows the confusion foreground fits for each binary evolution variation (different colored lines) and for each binary fraction model where the solid lines show FZ models and dotted lines show F50 models. The bottom panel shows the ratio of the number of DWDs orbiting in the LISA frequency band for the FZ models vs the F50 models for each variation. Even though the number of DWDs in the LISA band spans over two orders of magnitude, the ratio of FZ to F50, as well as the spectral shape of the confusion fit, stays fixed at a constant factor of two reduction. This suggests that assuming a metallicity-dependent binary fraction reduces the size of the Galactic close DWD population by a factor of $\sim$ two and the strength of the Galactic DWD GW foreground for LISA \emph{regardless of the chosen binary evolution model}.

\section{Conclusions}\label{sec:conclusions}
In this study, we have investigated the effects of assuming a metallicity-dependent binary fraction on the formation and evolution of the Galactic population of DWDs with a special focus on the implications for LISA. Based on our synthetic Milky-Way-like galaxy catalogs of DWDs, we find that applying a metallicity-dependent binary fraction changes the formation efficiency and evolutionary history of DWD populations. However, when considering the close DWD populations observable by LISA, we find that the only distinguishing features between models which assume a metallicity-dependent binary fraction (model FZ) and models which assume a flat $50\%$ binary fraction (model F50) are the population sizes and the strength of the Galactic DWD GW foreground. Models which assume a metallicity-dependent binary fraction produce Galactic DWD populations that are reduced by a factor of two relative to the standard model assumptions. This reduction extends to the height of the confusion foreground in the LISA data stream. 

We extended our study to include three binary evolution assumption variations to investigate whether the DWD population reduction was robust to changes in assumptions for mass transfer stability and common envelope ejection efficiencies. While our binary evolution assumption variations change the size of the LISA-observable populations dramatically, the reduction in the size of the close Galactic DWD population and the height of the confusion foreground for models which assume a metallicity-dependent binary fraction is robust. An important consequence of a lower Galactic DWD confusion foreground is that relative to the total DWD population, more DWDs can be individually resolved because of the reduction in competing GW signals. While the number of DWDs radiating GWs in the LISA frequency band is reduced by a factor of two for model FZ relative to model F50, the number of resolved sources is less affected with a population-wide reduction of $14\%$. These results are far-reaching since the strength of the Galactic DWD confusion foreground has direct consequences on the detectability of all other LISA sources with small SNRs. An increase in resolution capability from the reduced confusion foreground can be extended to other galactic binaries that LISA will observe at these frequencies like those involving neutron stars and stellar-origin black holes, as well as other more exotic GW sources like merging supermassive black holes, extreme mass ratio inspirals, or cosmological GW backgrounds. Based on our results, we suggest that studies which employ fits to the confusion foreground based on population synthesis results consider reducing the strength of the Galactic foreground PSD by a factor of two.


\begin{acknowledgments}
The authors are grateful for helpful discussions with Carles Badenes, Christine Mazzola Daher, and the Gravitational Waves and Astronomical Data groups at the CCA.  S.T.\ was supported by an Undergraduate Student Research Award (USRA) at CITA from the Natural Sciences and Engineering Research Council of Canada (NSERC), Reference \# 498223. K.B.\ is grateful for support from the Jeffrey L. Bishop Fellowship. The Flatiron Institute is supported by the Simons Foundation.
\end{acknowledgments}


\section*{Data Availability}

All data and software required to reproduce our results will be shared through GitHub and Zenodo on acceptance of this manuscript.

\software{\texttt{astropy} \citep{astropy:2013, astropy:2018}; 
          \cosmic\ \citep{Breivik2020a};
          \legwork\ \citep{Wagg2021};
          \texttt{matplotlib}\ \citep{matplotlib}; 
          \texttt{numpy}\ \citep{numpy}; 
          \texttt{pandas}\ \citep{mckinney-proc-scipy-2010, reback2020pandas}; 
          \texttt{scipy}\ \citep{scipy}
          }
          
\bibliographystyle{aasjournal}
\bibliography{bib}{}

\end{document}
